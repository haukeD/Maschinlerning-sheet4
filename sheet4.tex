%inizative bildverarbeitung
%llncs    article
\documentclass[a4paper,parskip=full-]{article}

\usepackage{amsmath}
\usepackage{amssymb}

\usepackage{subcaption}
\usepackage{float}
\restylefloat{figure}
\usepackage{graphicx}  % images
%\usepackage{qtree} % for trees
%auch für Bäume
%http://tex.stackexchange.com/questions/183866/tree-with-six-or-more-children
\usepackage{tikz, tikz-qtree}
\usepackage{lmodern}  %THE tex font :)
\usepackage{url}  %urls in references
%\usepackage{prettyref}
%\usepackage{pstricks} %graphicv2
\usepackage{cite}
\usepackage{enumerate}
\usepackage{multicol}
\usepackage{xspace}
\usepackage{color}
\usepackage{algorithmic} 
\usepackage{wasysym}
\usepackage[ngerman]{babel}
\usepackage[utf8]{inputenc} % Zeichenkodierung   utf8   latin1
%\include{biblio} % references
\usepackage{listings}                           % for source code inclusion
\usepackage{multirow} 
\usepackage{caption}
\usepackage{wrapfig}
%absatz
\usepackage{setspace} 

%breaking math formulars automaticly
%http://tex.stackexchange.com/questions/3782/how-can-i-split-an-equation-over-two-lines
\usepackage{breqn}

%für kurze Enumarates wie i,I a, A etc.
\usepackage[shortlabels]{enumitem}

%Durchstreichungen
%\cancel
%http://de.wikibooks.org/wiki/LaTeX-Kompendium:_F%C3%BCr_Mathematiker#Durchstreichungen
\usepackage{cancel}

%Für Römische Zahlen
\usepackage{enumitem}
%\usepackage{romannum}%stdclsdv

%Durchstreicen möglich
\usepackage[normalem]{ulem}

%Bessere Brüche
\usepackage{nicefrac}

%bookmarks
%\usepackage[pdftex,bookmarks=true]{hyperref}
%[pdftex,bookmarks=true,bookmarksopen,bookmarksdepth=2]
\usepackage{hyperref}
%\usepackage{scrhack}

%fußnoten
\usepackage{footnote}
\usepackage{caption} 

\usepackage{geometry}
\geometry{verbose,a4paper,tmargin=25mm,bmargin=25mm,lmargin=15mm,rmargin=20mm}

%randnotiz
\newcommand\mpar[1]{\marginpar {\flushleft\sffamily\small #1}}
\setlength{\marginparwidth}{3cm}

%svg Grafiken
%http://tex.stackexchange.com/questions/122871/include-svg-images-with-the-svg-package
%\usepackage{svg}

\usepackage{pgf}

%http://tex.stackexchange.com/questions/48653/making-subsections-be-numbered-with-a-b-c
\usepackage{chngcntr}
\counterwithin{subsection}{section}

%Sektions nicht Nummerrrieren (<=> section*{...})
% \makeatletter
% \renewcommand\@seccntformat[1]{}
% \makeatother
\setcounter{secnumdepth}{0}

\title{Machine Learning \\
Exercise sheet 3}

\author{Gruppe 9: \\Hauke Wree and Fridtjof Schulte Steinberg}

\newcommand{\R}[0]{{\mathbb{R}}}

\newcommand{\N}[0]{{\mathbb{N}}}
\newcommand{\C}[0]{{\mathbb{C}}}
\newcommand{\K}[0]{{\mathbb{K}}}
\newcommand{\lF}[0]{{\mathcal{l}}}
\newcommand{\I}[0]{{\mathcal{I}}}
\newcommand{\nhnh}[0]{{\frac{n}{2} \times \frac{n}{2}}} %nice
\newcommand{\norm}[1]{\left\lVert#1\right\rVert}
%\newcommand{\rm}[1]{\romannumeral #1}
\newcommand{\RM}[1]{\MakeUppercase{\romannumeral #1{.}}} 

\renewcommand \thesubsection{\alph{subsection}}

\begin{document}

\maketitle

\section{Exercise 1 (Decision trees for Boolean functions):}

\subsection{a)}


\begin{figure}[H]
    \centering
    \begin{subfigure}[b]{0.4\textwidth}

\begin{tabular}{|c|c|c|}
\hline
A & B & $A \land \neg B$ \\
\hline
0 & 0 & 0 \\
\hline
0 & 1 & 0 \\
\hline
1 & 0 & 1 \\
\hline
1 & 1 & 0 \\
\hline
\end{tabular}

\end{subfigure}
    \hfill
    \begin{subfigure}[b]{0.4\textwidth}

\begin{tikzpicture}[every tree node/.style={draw,circle},
   level distance=1.25cm,sibling distance=1cm,
   edge from parent path={(\tikzparentnode) -- (\tikzchildnode)}]
\Tree
[.A
    \edge node[auto=right,pos=.6] {0};
    [.no ]
    \edge node[auto=right,pos=.6] {1};
    [.B 
        \edge node[auto=right,pos=.8] {0};
        [.yes ]
        \edge node[auto=right,pos=.8] {1};
        [.no ]
    ]
]
\end{tikzpicture}
\end{subfigure}

\end{figure}

%b)
\subsection{b)}

\begin{figure}[H]
    \centering
    \begin{subfigure}[b]{0.4\textwidth}

\begin{tabular}{|c|c|c|c|c|}
\hline
A & B & C & $B \land C$  & $A \lor [B \land B$] \\
\hline
0 & 0 & 0 & 0 & 0\\
\hline
0 & 0 & 1 & 0 & 0\\
\hline
0 & 1 & 0 & 0 & 0\\
\hline
0 & 1 & 1 & 1 & 1\\
\hline
1 & 0 & 0 & 0 & 1\\
\hline
1 & 0 & 1 & 0 & 1\\
\hline
1 & 1 & 0 & 0 & 1\\
\hline
1 & 1 & 1 & 1 & 1\\
\hline
\end{tabular}

\end{subfigure}
    \hfill
    \begin{subfigure}[b]{0.4\textwidth}

\begin{tikzpicture}[every tree node/.style={draw,circle},
   level distance=1.25cm,sibling distance=1cm,
   edge from parent path={(\tikzparentnode) -- (\tikzchildnode)}]
\Tree
[.A
    \edge node[auto=right] {$0$};
    [.B 
       \edge node[midway,left] {$0$};
       [.no ]
       \edge node[midway,right] {$1$};
       [.C 
         \edge node[midway,left] {$0$};
         [.no ]
         \edge node[midway,right] {$1$};
         [.yes ]
       ]
        ]
    \edge node[auto=left] {$1$};
    [.yes
        ]
]
\end{tikzpicture}

\end{subfigure}

\end{figure}

%c)
\subsection{c)}

\begin{figure}[H]
    \centering
    \begin{subfigure}[b]{0.4\textwidth}

\begin{tabular}{|c|c|c|}
\hline
A & B & $A XOR B$ \\
\hline
0 & 0 & 0 \\
\hline
0 & 1 & 1 \\
\hline
1 & 0 & 1 \\
\hline
1 & 1 & 0 \\
\hline
\end{tabular}

\end{subfigure}
    \hfill
    \begin{subfigure}[b]{0.4\textwidth}

\begin{tikzpicture}[every tree node/.style={draw,circle},
   level distance=1.25cm,sibling distance=1cm,
   edge from parent path={(\tikzparentnode) -- (\tikzchildnode)}]
\Tree
[.A
    \edge node[auto=right] {$0$};
    [.B 
       \edge node[midway,left] {$0$};
       [.no ]
       \edge node[midway,right] {$1$};
       [.yes ]
        ]
    \edge node[auto=left] {$1$};
    [.B 
        \edge node[midway,left] {$0$};
        [.yes ]
        \edge node[midway,right] {$1$};
        [.no ]
        ]
]
\end{tikzpicture}

\end{subfigure}

\end{figure}

%d)
\subsection{d)}

\begin{figure}[H]
    \centering
    \begin{subfigure}[b]{0.4\textwidth}

\begin{tabular}{|c|c|c|c|c|c|c|}
\hline
A & B & C & D & $A \land B$ & $C \land D$ & $[A \land B] \lor [C \land D]$ \\
\hline
0 & 0 & 0 & 0 & 0 & 0 & 0 \\
\hline
0 & 0 & 0 & 1 & 0 & 0 & 0  \\
\hline
0 & 0 & 1 & 0 & 0 & 0 & 0 \\
\hline
0 & 0 & 1 & 1 & 0 & 1 & 1 \\
\hline
0 & 1 & 0 & 0 & 0 & 0 & 0 \\
\hline
0 & 1 & 0 & 1 & 0 & 0 & 0 \\
\hline
0 & 1 & 1 & 0 & 0 & 0 & 0 \\
\hline
0 & 1 & 1 & 1 & 0 & 1 & 1 \\
\hline
1 & 0 & 0 & 0 & 0 & 0 & 0 \\
\hline
1 & 0 & 0 & 1 & 0 & 0 & 0 \\
\hline
1 & 0 & 1 & 0 & 0 & 0 & 0 \\
\hline
1 & 0 & 1 & 1 & 0 & 1 & 1 \\
\hline
1 & 1 & 0 & 0 & 1 & 0 & 1 \\
\hline
1 & 1 & 0 & 1 & 1 & 0 & 1 \\
\hline
1 & 1 & 1 & 0 & 1 & 0 & 1 \\
\hline
1 & 1 & 1 & 1 & 1 & 1 & 1 \\
\hline

\end{tabular}

\end{subfigure}
    \hfill
    \begin{subfigure}[b]{0.4\textwidth}

\begin{tikzpicture}[every tree node/.style={draw,circle},
   level distance=1.25cm,sibling distance=1cm,
   edge from parent path={(\tikzparentnode) -- (\tikzchildnode)}]
\Tree
[.A
    \edge node[auto=right] {$0$};
    [.C 
       \edge node[midway,left] {$0$};
       [.no ]
       \edge node[midway,right] {$1$};
       [.D 
         \edge node[midway,left] {$0$};
         [.no ]
         \edge node[midway,right] {$1$};
         [.B 
          \edge node[midway,left] {$0$};
          [.no ]
          \edge node[midway,right] {$1$};
          [.yes ]
         ]
       ]
        ]
    \edge node[auto=left] {$1$};
    [.yes
        ]
]
\end{tikzpicture}

\end{subfigure}

\end{figure}

\section{Exercise 2 (Decision tree learning)}

\subsection{a)}

\subsubsection{i.}

\begin{tikzpicture}[every tree node/.style={draw,circle},
   level distance=2.25cm,sibling distance=2cm,
   edge from parent path={(\tikzparentnode) -- (\tikzchildnode)}]
\Tree
[.{Deadline ?}
    \edge node[auto=right] {Urgent};
    [.{Party ?} 
       \edge node[midway,left] {no};
       [.Study ]
       \edge node[midway,right] {yes};
       [.Party ]
    ]
    \edge node[auto=left] {Near};
    [.{Party ?} 
        \edge node[midway,left] {no};
        [.Study ]
        \edge node[midway,right] {yes};
        [.Party ]
    ]
    \edge node[auto=left] {None};
    [.{Party ?} 
        \edge node[midway,left] {no};
        [.Study ]
        \edge node[midway,right] {yes};
        [.Party ]
    ]
]
\end{tikzpicture}


\subsubsection{ii.}

\begin{tikzpicture}[every tree node/.style={draw,circle},
   level distance=2.25cm,sibling distance=2cm,
   edge from parent path={(\tikzparentnode) -- (\tikzchildnode)}]
\Tree
[.{Party ?}
    \edge node[auto=right] {No};
    [.{Deadline ?} 
       \edge node[midway,left] {Urgent};
       [.Study ]
       \edge node[midway,right] {Near};
       [.Study ]
       \edge node[midway,right] {None};
        [.Study ]
    ]
    \edge node[auto=left] {Yes};
    [.{Deadline ?} 
        \edge node[midway,left] {Urgent};
        [.Party ]
        \edge node[midway,right] {Near};
        [.Party ]
        \edge node[midway,right] {None};
        [.Party ]
    ]
]
\end{tikzpicture}

\subsubsection{Compare both decision trees.}

Der Wert der Deadline ist für das Ergebnis irrelevant nur der Wert von \textit{Is there a party?} ist für das Ergebnis relevant, so dass der 
Entscheidungsbaum auch so sein kann.

\begin{tikzpicture}[every tree node/.style={draw,circle},
   level distance=2.25cm,sibling distance=2cm,
   edge from parent path={(\tikzparentnode) -- (\tikzchildnode)}]
\Tree
[.{Party ?}
    \edge node[auto=right] {No};
    [.Study ]
    \edge node[auto=left] {Yes};
    [.Party ]
]
\end{tikzpicture}

\end{document}